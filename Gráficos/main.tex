\documentclass[12pt,a4paper]{article}

\usepackage[utf8]{inputenc}
%Pacote para utilizar os caracteres em português, inclusive com os acentos

\usepackage[T1]{fontenc}
\usepackage[brazil]{babel}
%Pacote para inserir o idioma do texto e configurar alguns comandos - Pelo jeito o último idioma tem preferência.

\def\Year{\expandafter\YEAR\the\year}
\def\YEAR#1#2#3#4{#1#2#3#4}
%As duas linhas acima definem o comando \Year que permite usar automaticamente o ano que está sendo compilado o arquivo. Útil para capa dos trabalhos e fontes de figuras e tabelas geradas pelo autor.

%\usepackage{pslatex}
\usepackage{times}
%Os dois pacotes acima alteram o tipo de letra para Times New Roman

\usepackage{amsmath,amsfonts,amssymb}
%Os dois pacotes acima são da American Mathematical Society para utilizar símbolo matemáticos no texto. O último evita a compactação da fração.

\usepackage{indentfirst}
% Pacote para colocar o avanço da primeira linha do parágrafo

\usepackage{pgfgantt}

\usepackage{pgfplots} %Permite criar gráficos em Latex
\pgfplotsset{compat=1.5} %This will configure the compatibility layer.You should have at leastcompat=1.3. The suggested value is printed to the.logfile after running TEX.
\usepackage{pgfplotstable} %pacote para importar dados para o latex
\usepackage{filecontents} %Permite importar dados em arquivos .CSV para Latex


\usepackage{graphicx,color}
%Pacote para inserir Figures e colorir o texto

\usepackage{amsmath} \numberwithin{figure}{section}	
%Este pacote é para numerar Figures com mais de um número

\usepackage[a4paper, top=3cm, bottom=2cm, left=3cm, right=2cm]{geometry}
%Pacote para formatar as margens do texto

\usepackage{subfig} %Permite colocar várias figuras lado a lado, com numeração automática de Figura "a", Figura "b"...
\usepackage{setspace}
%Pacote para alterar o espaçamento entre linhas

\linespread{1.5}
%Comando para colocar o espaçamento entre linhas no documento inteiro - Não utilizar unidade



%======Importação dos Dados para o Latéx =============


\begin{filecontents*}{dados.csv}
DiasEfDQO,EfDQOR1,EfDQOR2,IndDescarte,VazDescarte,CapTrat,Freq,NumPSM,CapMedTrat
1,96.6458333333333,90.821052631579,Petroquímica,13.7,$<$1.000,54,223,12
2,83.890243902439,97.65625,Ger.Energia,4.5,5.000,210,130,11
3,93.8720930232558,84.7317073170732,Siderúrgica,4,10.000,109,52,20
4,86.6595744680851,89.9166666666667,Têxtil,8,20.000,115,32,9
5,99.7738095238095,86.63,Papel,11.2,30.000,46,6,38
6,86.7441860465116,82.7244897959184,Aliment.,4.2,40.000,16,44,5
7,93.76,92.6666666666667,Mineração,11,50.000,17,19,9
9,84.6966292134831,87.8414634146342,Outras,7.8,100.000,13,74,7
10,80.7682926829268,85.72,,,,,,
11,89.75,92.8823529411765,,,,,,
12,90.71875,85.8260869565217,,,,,,
14,83.7040816326531,89.7857142857143,,,,,,
17,83.7684210526316,96.7,,,,,,
20,94,83.8674698795181,,,,,,
25,97.3,99.5,,,,,,
28,98.2,99.4,,,,,,
\end{filecontents*}

\pgfplotstableread[col sep=comma]{dados.csv}\datatable
\makeatletter
\pgfplotsset{
	/pgfplots/flexible xticklabels from table/.code n args={3}{%
		\pgfplotstableread[#3]{#1}\coordinate@table
		\pgfplotstablegetcolumn{#2}\of{\coordinate@table}\to\pgfplots@xticklabels
		\let\pgfplots@xticklabel=\pgfplots@user@ticklabel@list@x
	}
}
\makeatother
\begin{document}

\setlength{\parindent}{0pt} %para nao ter identação nos paragrafos
\setlength{\parskip}{1.5ex} %para espaçamento entre paragrafos
\section{Exemplos de gráficos de linhas}

\begin{figure}[!h]
	\centering
	\caption{Eficiência de remoção de DQO dos reatores}
	\label{DQO reatores}
	\begin{tikzpicture}
	\begin{axis}[legend entries={Reator 1, Reator 2},
	width=0.7\linewidth, %Define a largura do gráfico.
	height=7cm, %Se quiser alterar a altura do gráfico.
	xmin=0,
	%xmax=25, %limite máximo do eixo x
	axis x line=bottom, 
	axis y line=left, %por padrão, o tikz faz uma caixa (box) com linhas não só nos eixos x e y, mas também na direita e em cima. Estes dois comandos anteriores fazem ficar só os eixos.
	ymin=0,
	ymax=100,
	%xtickmin=0,
	%xtickmax=20, %Defini o menor e maior valor do eixo. Ajuda a distribuir os valores
	%xtick={0,5,...,20}, %Define quantos valores aparecem no eixo x 	
	ytick={0,10,...,100},
	%minor x tick num=2, %Se precisar colocar marcas adicionais
	legend style={
		at={(0.8,-0.25)}, %posição da legenda
		draw=none, %remove as linhas em volta da legenda
		%anchor=north,
		legend columns=2, %número de colunas na legenda
		%cells={anchor=west},
		%font=\footnotesize,
		%rounded corners=2pt,
	}, 
	%reverse legend, %legend pos=outer north east,
	xlabel=Tempo (d),
	ylabel=Eficiência de remoção de DQO (\%),
	]
	\addplot table [x=DiasEfDQO, y=EfDQOR1, col sep=comma,] {dados.csv}; %se quiser mudar a cor da linha só adicionar a cor assim: \addplot [black] table [x=...]
	\addplot table [x=DiasEfDQO, y=EfDQOR2, col sep=comma,] {dados.csv};
	\end{axis}
	\end{tikzpicture}
	
	\begin{center}
		{\footnotesize Fonte: do autor (\Year).} 
	\end{center}
\end{figure}

	
\newpage
%===============================================================================

Gráfico Profissional

\begin{figure}[!h]
	\centering
	\caption{Eficiência de remoção de DQO dos reatores}
	\label{DQOProfissional}

	
	
\end{figure}


\newpage
%===============================================================================


\section{Exemplos de gráficos de barras}

\begin{figure}[!h]
	\centering
	\subfloat[Descarte de águas residuárias de tipologias industriais na China]{
		\begin{tikzpicture}
		\begin{axis}[
		width=0.4\linewidth, %Define a largura do gráfico.
		ybar,
		enlargelimits=0.15,
		legend style={at={(0.5,-0.2)},anchor=north,legend columns=-1},
		ylabel={Descarte de efluentes (milhões m$^3$/d)},
		%xlabel={Capacidade do tratamento (m$^3$/d)},
		flexible xticklabels from table={dados.csv}{IndDescarte}{col sep=comma},
		xticklabel style={text height=1.5ex}, % To make sure the text labels are nicely aligned
		xtick=data,
		axis x line*=bottom, 
		axis y line=left, %por padrão, o tikz faz uma caixa (box) com linhas não só nos eixos x e y, mas também na direita e em cima. Estes dois comandos anteriores fazem ficar só os eixos.
		%nodes near coords, %coloca os valores acima das barras
		ymin=0,
		ymax=15,
		nodes near coords align={vertical},
		x tick label style={rotate=45,anchor=east}, %Caso queira rotacionar os rótulos do eixo x
		]
		%\addplot [black, fill=gray] coordinates{(Petroquímica,13.7) (Ger. Energia,4.5) (Siderúrgica,4) (Têxtil,8) (Papel,11.2) (Aliment.,4.2) (Mineração,11) (Outras,7.8)};	
		\addplot table[x expr=\coordindex,y=VazDescarte, color=black, fill=gray]{\datatable};	
		\end{axis}
		
		\end{tikzpicture}
		
		
	}
	\qquad %Coloca as duas na mesma linha
	\subfloat[Histograma da capacidade das plantas de tratamento por membrana na China]{
		\begin{tikzpicture}
		\begin{axis}[ybar,enlargelimits=0.15,legend style={at={(0.5,-0.2)},anchor=north,legend columns=-1},
		width=0.4\linewidth, %Define a largura do gráfico.
		ylabel={Frequência},
		xlabel={Capacidade do tratamento (m$^3$/d)},
		%symbolic x coords={$<$1.000,5.000,10.000,20.000,30.000,40.000,50.000,100.000},
		flexible xticklabels from table={dados.csv}{CapTrat}{col sep=comma},
		xticklabel style={text height=1.5ex}, % To make sure the text labels are nicely aligned
		xtick=data,
		axis x line*=bottom, 
		axis y line=left, %por padrão, o tikz faz uma caixa (box) com linhas não só nos eixos x e y, mas também na direita e em cima. Estes dois comandos anteriores fazem ficar só os eixos.
		%nodes near coords, %coloca os valores acima das barras
		ymin=0,
		ymax=250,
		ytick={0,50,...,200},
		nodes near coords,
		nodes near coords align={vertical},
		x tick label style={rotate=45,anchor=east}, %Caso queira rotacionar os rótulos do eixo x
		]
		%\addplot [black, fill=gray]
		%coordinates{($<$1.000,54) (5.000,210) (10.000,109)(20.000,115) (30.000,46)(40.000,16)(50.000,17)(100.000,13)};
		\addplot table[x expr=\coordindex,y=Freq, color=black, fill=gray]{\datatable};
		\end{axis}
		
		\end{tikzpicture}
		
	}
	\\
	
	\subfloat[Número de plantas de PSM por indústria]{
		\begin{tikzpicture}
		\begin{axis}[
		width=0.4\linewidth, %Define a largura do gráfico.
		ybar,
		enlargelimits=0.15,
		legend style={at={(0.5,-0.2)},anchor=north,legend columns=-1},
		ylabel={Número de PSM},
		%xlabel={Capacidade do tratamento (m$^3$/d)},
		flexible xticklabels from table={dados.csv}{IndDescarte}{col sep=comma},
		xticklabel style={text height=1.5ex}, % To make sure the text labels are nicely aligned
		xtick=data,
		axis x line*=bottom, 
		axis y line=left, %por padrão, o tikz faz uma caixa (box) com linhas não só nos eixos x e y, mas também na direita e em cima. Estes dois comandos anteriores fazem ficar só os eixos.
		%nodes near coords, %coloca os valores acima das barras
		ymin=0,
		ymax=250,
		ytick={0,50,...,200},
		nodes near coords, %coloca os valores acima das barras
		nodes near coords align={vertical},
		x tick label style={rotate=45,anchor=east}, %Caso queira rotacionar os rótulos do eixo x
		]
		\addplot table[x expr=\coordindex,y=NumPSM, color=black, fill=gray]{\datatable};
		\end{axis}
		
		\end{tikzpicture}
		
		
	}
	\qquad %Coloca as duas na mesma linha
	\subfloat[Capacidade média de tratamento das plantas com membrana]{
		\begin{tikzpicture}
		\begin{axis}[ybar,enlargelimits=0.15,legend style={at={(0.5,-0.2)},anchor=north,legend columns=-1},
		width=0.4\linewidth, %Define a largura do gráfico.
		ylabel={Capacidade (milhares m$^3$/d)},
		%xlabel={Capacidade do tratamento (m$^3$/d)},
		flexible xticklabels from table={dados.csv}{IndDescarte}{col sep=comma},
		xticklabel style={text height=1.5ex}, % To make sure the text labels are nicely aligned
		xtick=data,
		axis x line*=bottom, 
		axis y line=left, %por padrão, o tikz faz uma caixa (box) com linhas não só nos eixos x e y, mas também na direita e em cima. Estes dois comandos anteriores fazem ficar só os eixos.
		%nodes near coords, %coloca os valores acima das barras
		ymin=0,
		ymax=45,
		ytick={0,10,...,40},
		nodes near coords,
		nodes near coords align={vertical},
		x tick label style={rotate=45,anchor=east}, %Caso queira rotacionar os rótulos do eixo x
		]
		\addplot table[x expr=\coordindex,y=CapMedTrat, color=black, fill=gray]{\datatable};
		\end{axis}
		
		\end{tikzpicture}
		
	}
	
	
\end{figure}

	\newpage

%===============================================================================



\end{document}